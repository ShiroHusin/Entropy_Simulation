\chapter{Conclusion}
\noindent
Simulation runs to demonstrate the evolution of 1s according to the rules show that the underlying matrix $M$ will get more homogeneous. This is indicated by the increase within the proxy entropy measurement $H$ which calculates the unique occurrences of $(1,0), (0,1)$ patterns. The feature $\alpha$ referred to as the Move probability is also influential with the final entropy measurement and simulation runs showing a more modest influence.   \par

\vspace{0.3cm}
\noindent
There was one bug that was discovered during the observation of simulation runs. The code shows a slight bias of choosing the vector direction associated with the "Top" direction or the vector $[0, -1]$. This is seen in \ref{fig:output_100} where after 600 runs the scattered colored pixels tend to the upwards direction. Currently, consultation from a better coder is needed to identify the cause. \par

\vspace{0.3cm}
\noindent
Finally, more plans are underway for the next version of the simulation. One way of tinkering with the rules is to allow for each cell within matrix $M$ to take 9 possible values of $0, 1, 2, 3, 4, 5, 6, 7, 8$. This is to match with real-life quantum-mechanical systems in which each particle can have multiple energy levels. In this case, 0 corresponds to the ground state and 8 represents the maximum amount of energy or quanta that each particle can take. \par

