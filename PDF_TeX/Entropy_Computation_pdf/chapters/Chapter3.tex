\chapter{How its made}
\section{How some alcoholic RTDs are made}
As mentioned in section 1.1. Alcoholic RTDs can consist of beverages like Beer, wine, pre-mixed  cocktails, and or alcopops/cider. Let's start with the most popular one which is Beer. 
In general, there are several steps involved in making beer. These steps are shown in the Block-flow diagram (BFD) below:
\begin{figure}[H]
    \centering
    \includegraphics[width=15cm, height=7.5cm]{images/Beer_BFD.png}
    \caption{General Block flow diagram for Beer making}
    \label{fig:beer}
\end{figure}
The steps of the beer-making process can be broken down as follows: 
\begin{enumerate}
  \item \textbf{Malting:} The first step in making beer is to convert raw grains, such as barley, wheat, or rice, into malt, which is the primary ingredient in beer. This process, called malting, involves soaking the grains in water to trick them into thinking that they can germinate. The germination process is stopped by drying the grains.
  \item \textbf{Mashing:} After the grains have been malted, they are then ground into a fine powder called grist. The grist is mixed with hot water in a large vessel called a mash tun, to form a sweet liquid called wort. The wort contains the sugars that will be fermented by the yeast to create alcohol.
  \item \textbf{Boiling: }The wort is then transferred to a large kettle called a brew kettle and brought to a boil. During the boiling process, hops are added to the wort to impart bitterness, flavor, and aroma to the beer. The hops also act as a natural preservative and boiling also sterilize the concentrated wort.
  \item \textbf{Cooling: } After boiling, the wort is cooled quickly to bring it to fermentation temperature.
  \item \textbf{Fermentation: }The cooled wort is then transferred to a fermentation tank and yeast is added to start the fermentation process. Yeast converts the sugars in the wort into alcohol and carbon dioxide, which will create the characteristic effervescence of the beer.
  \item \textbf{Aging: }After fermentation, the beer is typically aged for a period of time in order to allow the flavors to develop and to clarify the beer.
  \item \textbf{Filtering: }The beer is then filtered to remove any remaining yeast or sediment, in order to make the beer clear.
  \item \textbf{Carbonation and packaging: }After filtering, the beer is carbonated, either naturally by introducing sugar or additional yeast, or by injecting carbon dioxide under pressure. The beer is then packaged into bottles, cans, or kegs, and is ready to be shipped to the consumer. 
\end{enumerate}
Another prominent alcoholic RTD that is commonly brought is wine. This process of making wine can vary depending on the type of wine but in general wine-making follows these steps: 
\begin{enumerate}
  \item \textbf{Harvest:} The first step in making wine is to harvest the grapes when they are at their peak of ripeness. This is typically done by hand, although mechanization is also used in some regions.
  \item \textbf{Crusing and pressing:} After the grapes are harvested, they are then crushed and pressed to extract the juice. During crushing, the grapes are broken down to release the juice. Pressing is used to extract more juice from the grapes and sometimes remove the skins. This process will vary depending on whether red, white or rosé wine is being made.
  \item \textbf{Fermentation: }The juice, called must, is then transferred to fermentation tanks. Yeast is added to start the fermentation process where the juice is converted into alcohol and carbon dioxide. The fermentation process can vary depending on the type of grape and the desired final product.
  \item \textbf{Clarification: }After fermentation, the wine is clarified to remove any remaining solids. This process can include techniques such as racking, fining, and filtration.
  \item \textbf{Aging: }After clarification, the wine is aged in oak barrels or stainless steel tanks. During this time, the wine develops its unique flavors and aromas.
  \item \textbf{Blending and packaging: }After aging, the winemaker will taste the wine and decide which barrels or tanks to blend together to create the final product. The wine is then bottled, corked, and sealed, ready for shipment and consumption.
\end{enumerate}
\section{How some non-alcoholic RTDs are made}
As mentioned in section 1.1, non-alcoholic RTDs can consist of Juice drinks, Tea or Coffee, and Carbonated drinks. It can also range towards more exotic RTDs like sports drinks, Kombucha, Diary products, or packaged coconut water.  \par

\vspace{0.3cm}
To make an RTD juice drink,  one can select from a huge repertoire of fruits such as mango, guava, apple, watermelon, and many others. However, the general block flow diagram of making juice drinks is: 
\begin{figure}[H]
    \centering
    \includegraphics[width=12cm, height=10cm]{images/Juice_BFD.png}
    \caption{General Block flow diagram for juice RTDs}
    \label{fig:juice}
\end{figure}
Breaking down each step into greater detail: \begin{enumerate}
  \item \textbf{Washing and dicing: } The first step in making juice RTDs is to extract the pulp and the fruit juice by dicing and pressing it.
  \item \textbf{Pulping: } The fruit pulp, the juice, and the fibers are collected. Depending on regulations, the fruit component has to be larger than a certain percentage. 
  \item \textbf{Filtering: }Prior to filtering, the fruit juice extract is added with sugar syrup like sucrose to assist in providing flavor. Filtering of juice is done to remove the solids that are present within the fruit extract. 
  \item \textbf{Homogenization: } After filtration, the fruit extract is homogenized to make an even distribution of total dissolved solids and liquids within the product. 
  \item \textbf{Mixing: }Prior to mixing, the fruit is added with preservatives and or color or citric acid. Preservatives are added so that its shelf life is prolonged and acid is used to produce a balance between sugars and acid and make a consistent flavor.
  \item \textbf{Sterilization: }After mixing is done, the mixture is sterilized to ensure that no potential bacteria or fungi grow within the product. Some companies might use flash sterilization and some don't. To each their own.
  \item \textbf{Packaging: } Finally, the product is then packaged in sterilized bottles and ready to be shipped. 
\end{enumerate}
Finally, seeing that Process Partners also do Kombucha a tea drink made from fermented tea leaves. I decided to make some sort of research on the manufacture of this RTD. Essentially, Kombucha is a fermented tea drink that is made by combining tea, sugar, and a symbiotic culture of bacteria and yeast, also known as a SCOBY (Symbiotic Culture Of Bacteria and Yeast) which creates a fermentation process that results in the formation of a probiotic beverage. The general process flow for making kombucha is as follows:
\begin{enumerate}
  \item \textbf{Tea brewing: } The first step in making kombucha is to brew the tea. This is typically done by steeping black or green tea leaves in hot water for a period of time, depending on the desired strength of the tea.
  \item \textbf{Sweetening: } After the tea has been brewed, sugar is added to the tea to feed the yeast culture. The amount of sugar added will depend on the desired final product and the length of fermentation.
  \item \textbf{Cooling: }The tea is then cooled to room temperature to ensure that it is at a safe temperature to add the starter culture 
  \item \textbf{SCOBY addition: } The cooled tea is then poured into a fermenting container and the SCOBY culture is added. The SCOBY culture contains the symbiotic bacteria and yeast that will ferment the tea and create kombucha.
  \item \textbf{Fermentation: }The mixture is then left to ferment for a period of time, depending on the desired final product and the length of fermentation. The length of time depends on factors such as temperature, pH level, and the type of yeast and bacteria used.
  \item \textbf{Flavorings: }After fermentation, the kombucha can be flavored with different fruits, herbs, or spices.
  \item \textbf{Packaging: }The Kombucha is then bottled in sterilized cans or plastic bottles.
\end{enumerate}
\pagebreak
\section{Differing process flows}
\subsection{Kombucha}
The above BFD's present a general process flow for making Beer, wine, juice, and Kombucha. However, each region might have different ways to make the final RTD. \par

\vspace{0,3cm}
If we take Kombucha as an example, the tea that is used to make Kombucha is different for different parts of the world. To my knowledge, it seems that Black tea is often the main ingredient used to make Kombucha but in China, green tea is often used to make Kombucha. \par

\vspace{0,3cm}
In the sweetening process for Kombucha, some regions and companies associated with those regions might use honey, maple syrup, or jaggery (cane sugar) to sweeten the drink instead of normal white sugar. In terms of the flavorings process, local regions especially in southeast Asia might use local ingredients such as ginger or lemongrass to cater with the local demand. \par

\vspace{0,3cm}
Likewise, GDP can also play a role in predicting what sort of method is in use. A country/sector with lower GDP might not have the same process flow for making kombucha but instead, the process might be entirely homemade. Just like winemaking, some regions might use physical labor rather than a mechanical press or machinery to press grapes or brew the tea. \par

\vspace{0,3cm}
\subsection{Beer}
In terms of beer making, the basic process of producing beer remains the same. That is it follows the steps of Malting, Mashing, Boiling, and Fermenting. \par

\vspace{0,3cm}
However, for different parts of world beer making differs in the ingredients used. In Germany, under the Reinheitsgebot law, which was introduced by by Duke Wilhelm IV of Bavaria in the year 1516, is still enforced today. This means that only Barley, Hops, yeast and water can only be used to make German beer. On the other hand, grains such as wheat, rice, oat or rye are commonly found in other beer brewing process. \par

\vspace{0,3cm}
Some people might add local products into Beer making. Such as adding fruits or spices into the final product. The timing on when these things are added are not exact and depends on the manufacturer's flavor profile. Some add it during the fermentation phase, which allows the yeast to interact with the spices and fruits creating a deep complex flavor, others add it at the end of the beer making process. \par

\vspace{0.3cm}
There are other notable examples of differences of the beer making process. For instance, in Africa beer is made from locally sourced ingredients such as millet, sorghum and maize instead of the traditional barley and hops. In Asia, Beer is often made with rice producing rice beer with a highly different flavor. 

\subsection{Juice RTDs}
Just like with the manufacture of Kombucha or beer. The process of making juice-ready-to-drink (RTD) beverages remains generally the same. That is the process of sourcing the juice, cleaning, cutting, pressing, pasteurizing, packaging and shipping is similar across the world. However, depending on region and local taste, the process shown in \ref{fig:juice} might differ slightly. \par

\vspace{0.3cm}
For instance, in countries where sugar cane grows such as southeast Asia, the Caribbeans or Africa a concentrated sweet dark liquid of boiling sugarcane juice might be used instead of sugar syrup. Additionally, different regions may have different ways of pasteurising its final product, some might employ normal pasteurisation while others might use flash pasteurisation where the stream is brought to a higher temperature but for a much shorter length of time. This flash pasteurisation process is typically used for easily perishable products; products that spoils easily. 
\pagebreak

\section{A possible PFD for beer making}
\begin{figure}[H]
    \centering
    \includegraphics[width=20cm, height=15cm,angle=-90]{images/Beer_PFD.jpg}
    \caption{General PFD for Beer making}
    \label{fig:beer_PFD}
\end{figure}
\pagebreak

Tables that explain ambiguous stream in more detail is shown down below:
\begin{center}
\begin{tabular}{ |p{3cm}||p{3cm}||  }
 \hline
 \multicolumn{2}{|c|}{\textbf{Stream description}} \\
 \hline
Stream number& Description\\
 \hline
3 & Grist\\
8 & Mash tun\\
11& Wort\\
14& Fermentation mixture\\
17& Beer filtrate\\
18& Packaged RTD beer\\
 \hline
\end{tabular}
\end{center}
