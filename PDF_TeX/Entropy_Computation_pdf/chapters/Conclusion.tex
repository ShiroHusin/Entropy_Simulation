\chapter{Conclusion}
\noindent
Research on RTD drinks were made to answer the question on how RTD drinks are made in different parts of the world and what sort of RTD drinks are available. \par

\vspace{0.3cm}
\noindent
The RTD category can be mainly broken down into 2 sections alcoholic and non-alcoholic RTDs. Alcoholic RTDs include: Beer, Wine, Alcopolps, and cider. Non-alcoholic RTDs include: Juice drinks, Tea and Coffee beverages, Carbonated drinks, Sports drinks, Kombucha, Diary products and many more.  \par

\vspace{0.3cm}
\noindent
Answering the question on what sort of RTD drinks are available throughout the world requires a data-driven approach where the alcohol consumption per capita per year and GDP per capita are used as proxy measurements on the demand of RTDs. Countries with low levels of alcohol consumption per capita is unlikely to have a large repertoire of alcoholic RTDs. Similarly, countries with low GDP per capita is probably unvaried in its offerings of different types of non-alcoholic RTDs. \par

\vspace{0.3cm}
\noindent
The Process of making RTDs generally follow the same pattern. However, regional demand and law might dictate what sort of ingredients are used. For example, in Germany, only barley, hops, water and yeast are allowed for beer making. In other parts of the world, the Barley can be mixed with rice or other grains such as sorghum, maize or rye. 